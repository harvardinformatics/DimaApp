\documentclass[a4paper]{article}
\usepackage{Sweave}
\begin{document}

\title{}
\author{}

\maketitle


\begin{Schunk}
\begin{Sinput}
> require(reshape)
> getSampleName <- function(df) {
+     cnames <- as.character(colnames(df))
+     xl <- strsplit(cnames, split='_')
+     xl <- lapply(xl, function(x) x[1])
+     x <- sub('S', '', unlist(xl))
+     spls <- unique(x)
+     return(spls)
+ }
> modTMTsixplexLabelNames <- function(df) {
+     ## label names are 126, 127, ..., 131
+     ## data table is read and columns relabeled in ds_resources_21JAN17.Rnw:label=datamatricesnotnormalized
+     ## labels/colnames prefixed with sample identifiers: S227, S238, etc
+     ## modTMTsixplexLabelNames() removes sample prefix
+     colnames(df) <- sub('S.*_', '', colnames(df))
+     return(df)
+ }
> gplot_parallel <- function(hdf, df, contrast) {
+     ## slightly modified copy of gplot_hkprots()
+     ## parallelplots of abundances of sets of proteins
+     ## parameters:
+     # hdf: data.frame of housekeeping proteins
+     # df: data.frame sample data
+     # contrast: eg., 'TP5 vs TP4', to determine the scale
+     hdf <- modTMTsixplexLabelNames(hdf) # added D21217
+     xhdf <- namerows(hdf, col.name='Protein')
+     xhdf <- melt(xhdf, id.var = 'Protein')
+     colnames(xhdf) <- c('Protein', 'Samples', 'Abundance')
+ 
+     xmin <- min(df, na.rm=TRUE)
+     xmax <- max(df, na.rm=TRUE)
+     s <- deparse(substitute(hdf))
+     sl <- unlist(strsplit(s, '\\.'))[1]
+     sl <- unlist(strsplit(sl, 'k'))[2]
+ 
+     ## ===!!CAREFUL, fix this!!===
+     # legend.position = 'bottom' for parallelplots
+     # legend.position = 'right' for parallelplots
+     pg <- ggplot(xhdf, aes(Samples, Abundance, group=Protein, color=Protein)) + geom_line()
+     pg <- pg + theme(axis.text.x = element_text(angle = 0, hjust = 1), plot.title=element_text(color='blue', hjust=0.5))
+     ## NOTICE: removed abundance limit values (D21217)
+     #pg <- pg + scale_y_continuous(limits = c(xmin, xmax )) + labs(x='Time Points', size=8)
+     pg <- pg + labs(x='Development Stage', size=10)
+     pg <- pg + ggtitle(paste('', contrast, sep=' ')) 
+     pg <- pg + theme(legend.position = 'bottom', legend.text = element_text(size = 8, color='black'),
+                      axis.text.x = element_text(angle = 90, size=8),
+                      axis.title.x = element_text(size = 12), axis.title.y = element_text(size = 12))
+     
+     ## ===!!CAREFUL, fix this!!===
+     #print(pg) # for paralleplots
+     return(pg) # for multiplot
+ }
> lattice_barchart <- function(df) {
+     xdf <- namerows(df, col.name='Protein')
+     xdf <- melt(xdf, id.var='Protein')
+     v <- stringr::str_split_fixed(xdf$variable, '_', 2)
+     xdf <- data.frame(xdf, v)
+     res <- barchart(value ~ X2|Protein, groups=X1, data=xdf, scales=list(x=list(rot=90)), par.settings=list(superpose.polygon=list(col='lightgreen')))
+     #barchart(value ~ X2|Protein+X1, data=ma, scales=list(x=list(rot=90)), par.settings=list(superpose.polygon=list(col='blue')))
+     return(res)
+ }
> gplot_barplot <- function(df) {
+     ## parameters:
+     # df: data.frame sample data
+     v <- stringr::str_split_fixed(colnames(df), '_', 2) # this line and next moved D21217
+     title <- unique(v[,1])
+     df <- modTMTsixplexLabelNames(df) # added D21217
+     
+     xdf <- namerows(df, col.name='Protein')
+     xdf <- melt(xdf, id.var = 'Protein')
+     colnames(xdf) <- c('Protein', 'Samples', 'Abundance')
+         
+     #v <- stringr::str_split_fixed(xdf$Samples, '_', 2)
+     #title <- unique(v[,1])
+ 
+     pg <- ggplot(xdf, aes(Samples, Abundance)) + geom_bar(stat='identity', fill='lightblue', alpha=1)
+     pg <- pg + labs(x='Development Stage', size=10)
+     pg <- pg + facet_grid(. ~ Protein) + ggtitle(title)
+     pg <- pg + theme(axis.text.x = element_text(angle=90, hjust=1), plot.title=element_text(color='blue', hjust=0.5))
+     
+     return(pg) # for multiplot
+ }
> # D031417: v0 here, v1 in ds_analysis_21JAN17.Rnw, v2 (modified version of v1) below
> plotECDF <- function(hdf, df) {
+     # see ds_analysis_21JAN17.Rnw for how it is used
+     cnames <- as.character(colnames(df))
+     xl <- strsplit(cnames, split='_')
+     xl <- lapply(xl, function(x) x[1])
+     x <- sub('S', '', unlist(xl))
+     spls <- unique(x)
+     
+     pltl <- NULL
+     for (spl in spls) {
+         ixs <- grep(spl, colnames(df))
+         ld <- stack(df[, ixs])
+         xhk <- hdf[, ixs]
+ 
+         pdf(NULL)
+         dev.control(displaylist="enable")
+         par.orig <- par(mfrow = c(2, 3), oma=c(0, 0, 3, 0))
+         for (ch in as.character(unique(ld$ind))) {
+             ldvals <- ld[ld$ind == ch,]$values
+             f <- ecdf(ldvals)
+             
+             clm <- which(colnames(xhk)==ch)
+             hkvals <- xhk[, clm]
+             names(hkvals) <- rownames(xhk)
+             hkvals <<- hkvals[!is.na(hkvals)]
+             p <- f(as.numeric(hkvals))
+            
+             protcoordl <- list()
+             for (prot in names(hkvals)) {
+                 protcoordl[[prot]] <- c(as.numeric(hkvals[prot]), 0, as.numeric(hkvals[prot]), f(as.numeric(hkvals[prot])))
+             }
+             
+             main <- unlist(strsplit(ch, split='_'))[2]
+             plot(f, main=main, xlab='Protein Expression (Abundance)', ylab='Cumulative Expression', col.main='orange') # turquoise
+             for (nm in names(protcoordl)) {
+                 lines(list(x=c(protcoordl[[nm]][1], protcoordl[[nm]][3]), y=c(protcoordl[[nm]][2], protcoordl[[nm]][4])), col='blue')
+                 jcoords <- jitter(protcoordl[[nm]][3:4], factor=1, amount=0.08)
+                 text(x=jcoords[1], y=jcoords[2], labels=nm, col='blue', cex=0.7, adj=c(0,0))
+                 title(paste('Sample', spl), outer=TRUE, col.main='lightgreen', cex.main=1.5)
+             }
+         }
+         par(par.orig)
+         plt <- recordPlot()
+         invisible(dev.off())
+         splname <- paste('s', spl, sep='')
+         pltl[[splname]] <- plt
+     }
+     return(pltl)
+ }
> # modified D031417:
> # v2 plot directly without saving
> # remove lines and replace by points
> plotECDF_v2 <- function(hdf, df, spl) {
+     df <- df[,grep(spl, colnames(df), ignore.case=TRUE)]
+     ld <- stack(df)
+ 
+     #par.orig <- par(mfrow = c(2, 3), oma=c(0, 0, 3, 0))
+     for (ch in as.character(unique(ld$ind))[1]) {
+         ldvals <- ld[ld$ind == ch,]$values
+         f <- ecdf(ldvals)
+         
+         clm <- which(colnames(hdf)==ch)
+         hkvals <- hdf[, clm]
+         names(hkvals) <- rownames(hdf)
+         hkvals <<- hkvals[!is.na(hkvals)]
+         p <- f(as.numeric(hkvals))
+            
+         protcoordl <- list()
+         for (prot in names(hkvals)) {
+             protcoordl[[prot]] <- c(as.numeric(hkvals[prot]), 0, as.numeric(hkvals[prot]), f(as.numeric(hkvals[prot])))
+         }
+         
+         main <- unlist(strsplit(ch, split='_'))[2]
+         plot(f, main=main, xlab='Protein Expression (Abundance)', ylab='Cumulative Expression', col.main='orange') # turquoise
+         for (nm in names(protcoordl)) {
+             points(list(x=c(protcoordl[[nm]][1]), y=c(protcoordl[[nm]][4])), col='blue', pch=20)
+             jcoords <- jitter(protcoordl[[nm]][3:4], factor=1, amount=0.08)
+             #title(paste('Sample', spl), outer=TRUE, col.main='lightgreen', cex.main=1.5)
+         }
+     }
+ }
> # this is a copy from ds_analysis_A_27FEB17.Rnw:label=plotresults (D21517)
> addAnnotTt <- function(tt) {
+     # tt: output from topTable() with fixed contrast, eg., c1
+     # contrast: eg., T5vsT4 for title
+     # annot: either 'gene' or 'acc' (or some other string)
+     pname <- unlist(mget(rownames(tt), eacc2name, ifnotfound=unlist(mget(rownames(tt), eBBacc2name, ifnotfound=rownames(tt)))))
+     df <- data.frame(Name=pname,tt)
+     genesym <- unlist(mget(rownames(tt), eacc2sym, ifnotfound=unlist(mget(rownames(tt), eBBacc2sym, ifnotfound=rownames(tt)))))
+     df <- data.frame(Symbol=genesym, df)
+     df$logFC <- ifelse(df$logFC < 0, -2^(-df$logFC), 2^df$logFC)
+     df <- df[, c(1:3,7)]
+     colnames(df) <- c('Symbol', 'Name', 'FC', 'pval')
+     return(df)
+ }
> ## not done (D22717)
> addAnnotTt_lnk<- function(tt) {
+     # tt: output from topTable() with fixed contrast, eg., c1
+     # contrast: eg., T5vsT4 for title
+     # annot: either 'gene' or 'acc' (or some other string)
+     pname <- unlist(mget(rownames(tt), eacc2name, ifnotfound=unlist(mget(rownames(tt), eBBacc2name, ifnotfound=rownames(tt)))))
+     df <- data.frame(Name=pname,tt)
+     genesym <- unlist(mget(rownames(tt), eacc2sym, ifnotfound=unlist(mget(rownames(tt), eBBacc2sym, ifnotfound=rownames(tt)))))
+     # doesn't work
+     genesym <- paste('www.genecards.org/cgi-bin/carddisp.pl?gene',genesym, sep='=')
+     # something like this
+     #doc <- tags$html(tags$body(a(href="http://www.lalala.com"))
+     df <- data.frame(Symbol=genesym, df)
+     df$logFC <- ifelse(df$logFC < 0, -2^(-df$logFC), 2^df$logFC)
+     df <- df[, c(1:3, 7)]
+     colnames(df) <- c('Symbol', 'Name', 'FC', 'pval')
+     return(df)
+ }
> #mss.df <- as.data.frame(exprs(mss))
> #hk.df <- mss.df[rownames(mss.df) %in% hk,]
> #plotECDF(hk.df, mss.df)
> 
> 
> ## ==== This function taken from the web ====
> # Multiple plot function
> #
> # ggplot objects can be passed in ..., or to plotlist (as a list of ggplot objects)
> # - cols:   Number of columns in layout
> # - layout: A matrix specifying the layout. If present, 'cols' is ignored.
> #
> # If the layout is something like matrix(c(1,2,3,3), nrow=2, byrow=TRUE),
> # then plot 1 will go in the upper left, 2 will go in the upper right, and
> # 3 will go all the way across the bottom.
> #
> multiplot <- function(..., plotlist=NULL, file, cols=1, layout=NULL) {
+     library(grid)
+ 
+     # Make a list from the ... arguments and plotlist
+     plots <- c(list(...), plotlist)
+ 
+     numPlots = length(plots)
+ 
+     # If layout is NULL, then use 'cols' to determine layout
+     if (is.null(layout)) {
+         # Make the panel
+         # ncol: Number of columns of plots
+         # nrow: Number of rows needed, calculated from # of cols
+         layout <- matrix(seq(1, cols * ceiling(numPlots/cols)),
+                          ncol = cols, nrow = ceiling(numPlots/cols))
+     }
+ 
+     if (numPlots==1) {
+         print(plots[[1]])
+ 
+     } else {
+         # Set up the page
+         grid.newpage()
+         pushViewport(viewport(layout = grid.layout(nrow(layout), ncol(layout))))
+ 
+         # Make each plot, in the correct location
+         for (i in 1:numPlots) {
+         # Get the i,j matrix positions of the regions that contain this subplot
+             matchidx <- as.data.frame(which(layout == i, arr.ind = TRUE))
+ 
+             print(plots[[i]], vp = viewport(layout.pos.row = matchidx$row,
+                                             layout.pos.col = matchidx$col))
+         }
+     }
+ }
> 
\end{Sinput}
\end{Schunk}
\end{document}
